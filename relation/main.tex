\documentclass[a4paper]{beamer}
% Preambolo
\newcommand{\me}{\small}    
\usepackage{tikz}
\usepackage{graphicx}
\usepackage[utf8]{inputenc}
\usepackage[T1]{fontenc}
\usepackage{lmodern}
\usepackage[italian]{babel}
\usepackage{xcolor}
\usepackage{amsmath}
\usepackage{amssymb}
\usepackage{hyperref} 

\graphicspath{{immagini_presentazione/}}
\newcommand*\circled[1]{\tikz[baseline=(char.base)]{\node[shape=circle,draw,inner sep=1pt] (char) {#1};}}
% ============================================================
%                   INTESTAZIONE E STILE
% ============================================================

\title{Integrazione di Navier-Stokes e applicazioni}
\author{\texorpdfstring{F.A.F. Antonacci}{F.A.F. Antonacci}}
\date{\today}

\setbeamercolor{palette primary}{bg=blue!70, fg=white}
\setbeamertemplate{section in toc}[circle]
\setbeamertemplate{navigation symbols}{}
\newcounter{mysection} 
\newcommand{\mysections}{1,2,3,4,5,6,7,8} 

\setbeamertemplate{headline}{%
  \begin{beamercolorbox}[wd=\paperwidth,ht=8mm,dp=2mm,leftskip=2mm,rightskip=10mm]{palette primary}%
    \tikz[baseline]{%
      \foreach \s [count=\i] in \mysections {
        \ifnum\i=\value{section}
          \fill[white] (\i*5mm,3mm) circle (3pt); 
        \else
          \draw[white!50] (\i*5mm,3mm) circle (3pt); 
        \fi
      }
      \node[anchor=east, white, font=\me]
        at (\paperwidth-3mm,3mm)
        {\insertsectionhead};
    }
  \end{beamercolorbox}%
}

\setbeamertemplate{footline}{%
  \leavevmode%
  \hbox to \paperwidth{%
    \hspace{0.02cm} \begin{tikzpicture}%
             \node  [opacity=0.5]{\includegraphics[height=7mm]{immagini_presentazione/unipopipopi_transparent.png}};%
  \end{tikzpicture}
    \hfill
    \large \insertframenumber{} / \inserttotalframenumber
    \hspace{0.2cm}
  }%
  \vskip 0.2cm
}

% ============================================================
%                   TITLE PAGE & INDICE
% ============================================================
\begin{document}

\frame{\titlepage}

\begin{frame}{Indice}
  \tableofcontents
\end{frame}


% ============================================================
%                   Integrazione di Navier stokes
% ============================================================

\section{Idea}

\begin{frame}
  \frametitle{Navier-Stokes comprimibile}  
  Ho assunto che il mio fluido non assorba né conduca calore, sia comprimibile e viscoso.
  \[
    \frac{\partial\vec{u}}{\partial t} = - (\vec{u} \cdot \nabla) \vec{u}- \frac{\nabla p}{\rho} 
    + \frac{\mu \nabla^2 \vec{u}}{\rho} 
    + \frac{(\zeta + \mu/3) \nabla (\nabla \cdot \vec{u})}{\rho} 
    + \frac{\vec{F}}{\rho}
  \]
\begin{itemize}
    \item \(\text{conv} = (\vec{u} \cdot \nabla) \vec{u}\) è il termine convettivo, che rappresenta il trasporto della velocità dal flusso stesso;
    \item \(- \nabla p / \rho\) è il gradiente di pressione, responsabile dell’accelerazione dovuta alla pressione;
    \item \(\mu \, \nabla^2 \vec{u}/\rho\) rappresenta la viscosità del fluido (diffusione della velocità);
    \item \((\zeta + \mu/3) \, \nabla (\nabla \cdot \vec{u}) / \rho\) è il termine viscosità bulk, che interviene se il fluido è comprimibile;
    \item \(\vec{F}/\rho\) è la forza esterna applicata sul fluido.
\end{itemize}

\end{frame}

\begin{frame}
  \frametitle{Incomprimibilità o politropica e continuità?}  
  Ho scelto di introdurre una politropica invece di imporre l'incomprimibilità per i seguenti motivi:
  \begin{itemize}
    \item supponendo processi adiabatici posso controllare $\kappa$ della legge $p=\kappa \rho^\gamma$;
    \item posso aumentare $\kappa$ in modo arbitrario per aumentare la velocità del suono fino a rendere il fluido incomprimibile;
    \item non ho voglia di introdurre la risoluzione dell'equazione del laplaciano per la pressione: introdurrei pure ulteriore rumore numerico;
    \item scrivo meno codice.
  \end{itemize}

  Ho utilizzato l'equazione di continuità:
       $$\frac{\partial \rho}{\partial t}+\nabla \cdot (\rho \vec{u})=0.$$
  Con le tre equazioni appena enunciate ho chiuso il sistema.
\end{frame}

\begin{frame}

    \frametitle{La discretizzazione}  
    \begin{columns}
      \begin{column}{0.5\textwidth}
        \begin{itemize}
          \item \textbf{Integratore euleriano}
          \item \textbf{Approccio euleriano}
          \begin{itemize}
            \item per aggiornare la densità uso continuità;
            \item per aggiornare il campo delle velocità uso Navier-Stokes;
            \item per determinare la pressione utilizzo la politropica.          
          \end{itemize}
        \end{itemize}
      \end{column}

      \begin{column}{0.5\textwidth}
        \begin{figure}
          \includegraphics[width=1\textwidth]{grid.png}
          \caption{Esempio di griglia usata nell'integrazione. Si è presa un'immagine ravvicinata per migliorarne la visibilità.}
        \end{figure}  
      \end{column}
    \end{columns}

\end{frame}


% ============================================================
%                   Convergenza
% ============================================================
\section{Convergenza}
\begin{frame}
  \frametitle{Convergenza della simulazione}  
  Le integrazioni numeriche divergono se il passo temporale non è 
  sufficientemente piccolo:
  desidero che la risoluzione temporale della mia simulazione sia più grande delle velocità caratteristiche del mio problema.
  \begin{itemize}
    \item $dx = 1$;
    \item $dt= CFL \frac{dx}{\left|\vec{u}\right|_{max}+c_{s,max}}$
    \item $c_s = \frac{\partial p}{\partial \rho}|_{s}= \sqrt{\gamma  p / \rho} $ 
           è la velocità locale del suono supposta la compressione adiabatica del fluido;
    \item   $CFL$ è un fattore di adimensionale convergenza della simulazione.

  \end{itemize}


\end{frame}

\begin{frame}
  \frametitle{Courant-Friedrichs-Lewy}  
  \begin{itemize}
    \item   La simulazione, siccome si basa su un integratore euleriano $\Rightarrow$ 
    \item   le grandezze integrate divergere come $CFL \propto 1/N$ (con $N$ numero di punti in cui si divide il passo temporale) a meno che non ci siano soluzioni stabili.
    \item   Il limite nella scelta di $CFL$ è il tempo di calcolo e la memoria del computer$\Rightarrow$  
    \item   per ovviare a questo si registra uno stato della griglia saltando un numero proporzionale a $CFL$.

  \end{itemize}

\end{frame}

% ============================================================
%                   Condizioni al contorno
% ============================================================

\section{Condizioni al contorno}
\begin{frame}
  \frametitle{Condizioni al controno}  
    \begin{itemize}
      \item condizioni al contorno sono deteminate nel codice dalla definizione delle derivate (\texttt{gradient()} di \texttt{numpy});
      \item il mezzo è assunto infinito e il prologato in maniera continua;
      \item $\Rightarrow$ Detta $\hat{n}$ la normale al bordo, si desidera:
          \begin{itemize}
            \item $(\hat{n} \cdot \nabla) \vec{u}\leq0$, 
            \item $(\hat{n} \cdot \nabla) \rho\leq0$;
          \end{itemize} 
      \item simulazioni sono valide finchè non si raggiunge il bordo;
      \item altrimenti la simulazione diverge.
    \end{itemize}
\end{frame}


% ============================================================
%                   Profilo gaussiano di densità
% ============================================================
\section{Profilo gaussiano di densità}

\begin{frame}
  \frametitle{Profilo gaussiano di densità}

  \centering
  \begin{figure}
    \href{run:/media/candido/Extreme SSD2/Unipi/Terzo anno/Fluidodinamica/relation/immagini_presentazione/gaussian1.mp4}{
      \includegraphics[width=0.5\textwidth]{immagini_presentazione/gaussian1.png}
    }
    \caption{Profilo gaussiano di densità in simmetria sferica che parte fermo e viene lasciato evolvere. Si marcano due isobare a due valori diversi.}
  \end{figure}

\end{frame}


% ============================================================
%                   Profilo gaussiano di densità 2
% ============================================================
\section{Variante del profilo gaussiano di densità}

\begin{frame}
  \frametitle{Variante del profilo gaussiano di densità}

    \centering
  \begin{figure}
  \href{file:/media/candido/Extreme SSD2/Unipi/Terzo anno/Fluidodinamica/relation/immagini_presentazione/gaussian2.mp4}{
    \includegraphics[width=0.6\textwidth]{immagini_presentazione/gaussian2.png}
    }
    \caption{Profilo gaussiano di densità in simmetria sferica, di cui una sezione ha una velocità iniziale. Si rappresentano un isobara e il campo delle velocità.}
  \end{figure}

\end{frame}
% ============================================================
%                   Onde sonore
% ============================================================
\section{Onda sonora}

\begin{frame}
  \frametitle{Onda sonora}    
      \centering
        \begin{figure}
              \href{file:/media/candido/Extreme SSD2/Unipi/Terzo anno/Fluidodinamica/relation/immagini_presentazione/planes.mp4}{
              \includegraphics[width=0.6\textwidth]{immagini_presentazione/planes.png}}
                \caption{Profilo gaussiano di densità in simmetria planare con velocità modulata gaussianamente e ortogonale al gradiente della densità.
                 Si rappresentano un isobara e il campo delle velocità.}
        \end{figure}
\end{frame}

% ============================================================
%                   Vortici
% ============================================================
\section{Vortici}
\begin{frame}
  \frametitle{Vortice lineare}
        \centering
        \begin{figure}
                \href{file:/media/candido/Extreme SSD2/Unipi/Terzo anno/Fluidodinamica/relation/immagini_presentazione/linvort.mp4}{
                  \includegraphics[width=0.6\textwidth]{immagini_presentazione/linvort.png}}
                \caption{Densità costante, campo delle velocità derteminato da $\vec{v}=\vec{\omega}\times \vec{r}$.
                Il fluido è viscido e comprimibile, tuttavia siamo nel limite in cui la velocità del suono è alta.}
        \end{figure}
\end{frame}


\begin{frame}
  \frametitle{Vortice di potenziale}
          \centering
        \begin{figure}
                \href{file:/media/candido/Extreme SSD2/Unipi/Terzo anno/Fluidodinamica/relation/immagini_presentazione/potvort.mp4}{
                  \includegraphics[width=0.6\textwidth]{immagini_presentazione/potvort.png}}
                \caption{Densità costante, campo delle velocità derteminato da $\vec{v}=\frac{\kappa \hat{\phi}}{r}$.
                Il fluido è viscido e comprimibile, tuttavia siamo nel limite in cui la velocità del suono è alta.}
        \end{figure}
\end{frame}

\begin{frame}
  \frametitle{Vortice di potenziale con viscosità nulla}
                    \centering
        \begin{figure}
                \href{file:/media/candido/Extreme SSD2/Unipi/Terzo anno/Fluidodinamica/relation/immagini_presentazione/potvortnu0.mp4}{
                  \includegraphics[width=0.6\textwidth]{immagini_presentazione/potvortnu0.png}}
                \caption{Densità costante, campo delle velocità derteminato da $\vec{v}=\frac{\kappa \hat{\phi}}{r}$.
                Il fluido è inviscido e comprimibile, tuttavia siamo nel limite in cui la velocità del suono è alta.}
        \end{figure}
\end{frame}
% ============================================================
%                   Jet
% ============================================================
\section{Jet puntiforme}
\begin{frame}
  \frametitle{Jet puntiforme: Trattazione analitica alla Landau}  
    Supponiamo che in fluido venga immessa della quantità di moto da una sorgente puntiforme con le seguenti ipotesi:
    \begin{itemize}
      \item fenomeno all'equilibrio (il che implica ad esempio $\nabla \cdot \vec{u}=0, \,\partial_t \rho=0$); 
      \item osservare da lontano la sorgente;
      \item in coordinate sferiche ($r,\theta,\phi$) le velocità siano ($u,v,w$);
      \item simmetria per rotazione $\Rightarrow$ $r,\theta$ sono le uniche variabili indipendenti, $w=0$;
      \item stream-function $\psi$  $\Rightarrow$ $u=\frac{1}{r^2 \sin \theta}\frac{\partial \psi}{\partial \theta}
                                    \,v=-\frac{1}{r \sin \theta}\frac{\partial \psi}{\partial \theta}$;
      \item per la conservazione del momento deve scalare come $r^{-1}$
      \item $\Rightarrow \psi=r \alpha f(\theta)$, con $\alpha$ costante.
    \end{itemize}
\end{frame}

\begin{frame}
  \frametitle{Soluzione analitica del Jet puntiforme}  
    Imponendo Navier-Stokes per $u$ e per $v$ con passaggi non ovvi si ottiene:
    \begin{itemize}
      \item $f=\frac{2 \sin^2\theta}{1+ c -\cos \theta}$
      \item $\psi=\frac{2 \alpha \sin^2\theta}{1+ c -\cos \theta}$
      \item $u=\frac{4\alpha(1+c-\cos \theta) \cos \theta-2\sin^2 \theta}{r(1+c-\cos\theta)^2}$
      \item $v=\frac{-2\alpha \sin^2 \theta}{r(1+c-\cos\theta)}$
    \end{itemize}
\end{frame}


\begin{frame}
  \frametitle{Plot della soluzione analitica}  
      \begin{figure}
        \includegraphics[width=0.50\textwidth]{anal_get2D.pdf}
        \caption{Sezione del campo delle velocità nel piano X-Z della soluzione analitica del Jet-puntiforme.}
      \end{figure}
\end{frame}

\begin{frame}
  \frametitle{Velocità costante}  
          \centering
        \begin{figure}
                \href{file:/media/candido/Extreme SSD2/Unipi/Terzo anno/Fluidodinamica/relation/immagini_presentazione/jet3.mp4}{
                    \includegraphics[width=0.6\textwidth]{immagini_presentazione/jet3.png}}
                \caption{Fluido con densità costante. Velocità al centro mantenuta costante e modulata con una gaussiana centrata e molto stretta.
                          Si rappresenta il campo delle velocità.}
        \end{figure}
\end{frame}

\begin{frame}
    \frametitle{Forza costante}  
  \begin{columns}
      \begin{column}{0.5\textwidth}      
        \begin{figure}
        \includegraphics[width=\textwidth]{jet2closeup.png}
      \end{figure}
    \end{column}
  \begin{column}{0.5\textwidth}
        \centering
        \begin{figure}
                \href{file:/media/candido/Extreme SSD2/Unipi/Terzo anno/Fluidodinamica/relation/immagini_presentazione/jet2.mp4}{
                    \includegraphics[width=0.8\textwidth]{immagini_presentazione/jet2.png}}
                \caption{Fluido con densità costante. Forza costante modulata con una gaussiana centrata e molto stretta.
                          Si rappresenta il campo delle velocità.}
        \end{figure}    
      \end{column}
  \end{columns}
\end{frame}

\begin{frame}
  \frametitle{Aggiustando il bordo...}  
        \centering
        \begin{figure}
                \href{file:/media/candido/Extreme SSD2/Unipi/Terzo anno/Fluidodinamica/relation/immagini_presentazione/jet3.mp4}{
                    \includegraphics[width=0.6\textwidth]{immagini_presentazione/jet3.png}}
                \caption{Fluido con densità costante. Forza costante modulata con una gaussiana centrata e molto stretta.
                          Si tenta di eliminare il flusso di ritorno imponendo che la velocità e la densità siano costanti al contorno.
                          Si rappresenta il campo delle velocità.}
        \end{figure}
\end{frame}

\begin{frame}
  \frametitle{Vediamo se la soluzione analitica è stabile numericamente.}  
        \centering
        \begin{figure}
                \href{file:/media/candido/Extreme SSD2/Unipi/Terzo anno/Fluidodinamica/relation/immagini_presentazione/jet4.mp4}{
                    \includegraphics[width=0.6\textwidth]{immagini_presentazione/jet4.png}}
                \caption{Evoluzione temporale della soluzione analitica del Jet puntiforme. 
                          E' stato necessario introdurre rendere la descrizione inesatta per evitare divergenze numeriche almeno all'inizio. 
                          Si rappresenta il campo delle velocità.}
        \end{figure}
\end{frame}

\begin{frame}
  \frametitle{Tuttavia allargando...}  
        \centering
        \begin{figure}
                \href{file:/media/candido/Extreme SSD2/Unipi/Terzo anno/Fluidodinamica/relation/immagini_presentazione/jet5.mp4}{
                    \includegraphics[width=0.6\textwidth]{immagini_presentazione/jet5.png}}
                \caption{Fluido con densità costante. Forza costante modulata con una gaussiana centrata e molto stretta.
                          Si tenta di eliminare il flusso di ritorno imponendo che la velocità e la densità siano costanti al contorno.
                          Si rappresenta il campo delle velocità.}
        \end{figure}
\end{frame}
\end{document}
